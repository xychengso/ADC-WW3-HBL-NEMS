\begin{footnotesize}
\begin{verbatim}
$ -------------------------------------------------------------------- $
$ WAVEWATCH III Point output post-processing                           $
$--------------------------------------------------------------------- $
$ First output time (yyyymmdd hhmmss), increment of output (s), 
$ and number of output times.
$
  19680606 060000  3600.  7
$
$ Points requested --------------------------------------------------- $
$ Define points for which output is to be generated. 
$
$ 1
$ 2
  3
$ 4
$
$ mandatory end of list
 -1
$
$ Output type ITYPE [0,1,2,3,4]
$
  1
$ -------------------------------------------------------------------- $
$ ITYPE = 0, inventory of file.
$            No additional input, the above time range is ignored.
$
$ -------------------------------------------------------------------- $
$ ITYPE = 1, Spectra.
$          - Sub-type OTYPE :  1 : Print plots.
$                              2 : Table of 1-D spectra
$                              3 : Transfer file.
$          - Scaling factors for 1-D and 2-D spectra Negative factor
$            disables, output, factor = 0. gives normalized spectrum.
$          - Unit number for transfer file, also used in table file
$            name.
$          - Flag for unformatted transfer file.
$
  1   0.  0.  33  F
$
$ The transfer file contains records with the following contents.
$
$ - File ID in quotes, number of frequencies, directions and points.
$   grid name in quotes (for unformatted file C*21,3I,C*30).
$ - Bin frequencies in Hz for all bins.
$ - Bin directions in radians for all bins (Oceanographic conv.).
$                                                         -+
$ - Time in yyyymmdd hhmmss format                         | loop
$                                             -+           |
$ - Point name (C*10), lat, lon, d, U10 and    |  loop     | over
$   direction, current speed and direction     |   over    |
$ - E(f,theta)                                 |  points   | times
$                                             -+          -+
$
$ The formatted file is readable using free format throughout.
$ This data set can be used as input for the bulletin generator
$ w3split.
$
$ -------------------------------------------------------------------- $
$ ITYPE = 2, Tables of (mean) parameter
$          - Sub-type OTYPE :  1 : Depth, current, wind
$                              2 : Mean wave pars.
$                              3 : Nondimensional pars. (U*)
$                              4 : Nondimensional pars. (U10)
$                              5 : 'Validation table'
$                              6 : WMO standard output 
$          - Unit number for file, also used in file name.
$
$   6  66
$
$ If output for one point is requested, a time series table is made,
$ otherwise the file contains a separate tables for each output time.
$
$ -------------------------------------------------------------------- $
$ ITYPE = 3, Source terms
$          - Sub-type OTYPE :  1 : Print plots.
$                              2 : Table of 1-D S(f).
$                              3 : Table of 1-D inverse time scales
$                                  (1/T = S/F).
$                              4 : Transfer file
$          - Scaling factors for 1-D and 2-D source terms. Negative
$            factor disables print plots, factor = 0. gives normalized
$            print plots.
$          - Unit number for transfer file, also used in table file
$            name.
$          - Flags for spectrum, input, interactions, dissipation,
$            bottom, ice and total source term.
$          - scale ISCALE for OTYPE=2,3
$                              0 : Dimensional.
$                              1 : Nondimensional in terms of U10
$                              2 : Nondimensional in terms of U*
$                             3-5: like 0-2 with f normalized with fp.
$          - Flag for unformatted transfer file.
$
$ 1  0. 0. 50   T T T T T T T 0  F
$
$ The transfer file contains records with the following contents.
$
$ - File ID in quotes, number of frequencies, directions and points,
$   flags for spectrum and source terms (C*21, 3I, 6L)
$ - Bin frequencies in Hz for all bins.
$ - Bin directions in radians for all bins (Oceanographic conv.).
$                                                         -+
$ - Time in yyyymmdd hhmmss format                         | loop
$                                             -+           |
$ - Point name (C*10), depth, wind speed and   |  loop     | over
$   direction, current speed and direction     |   over    |
$ - E(f,theta) if requested                    |  points   | times
$ - Sin(f,theta) if requested                  |           |
$ - Snl(f,theta) if requested                  |           |
$ - Sds(f,theta) if requested                  |           |
$ - Sbt(f,theta) if requested                  |           |
$ - Sice(f,theta) if requested                 |           |
$ - Stot(f,theta) if requested                 |           |
$                                             -+          -+
$ -------------------------------------------------------------------- $
$ ITYPE = 4, Spectral partitions and bulletins
$          - Sub-type OTYPE :  1 : Spectral partitions
$                              2 : Bulletins ASCII format
$                              3 : Bulletins CSV format
$                              4 : Bulletins ASCII and CSV formats
$          - Unit number for transfer file, also used in table file
$            name.
$          - Reference date/time in YYYYMMDD HHMMSS format, used for 
$            including in bulletin legend, and computing forecast time
$            in CSV type output (if the first field is negative, the 
$            reference time becomes the first simulation time slice)
$          - Three-character code indicating time zone (UTC, EST etc)
$
$   4  2 19680606 060000 'UTC'
$
$ The transfer file contains records with the following contents.
$
$ - File ID in quotes, number of frequencies, directions and points.
$   grid name in quotes (for unformatted file C*21,3I,C*30).
$ - Bin frequencies in Hz for all bins.
$ - Bin directions in radians for all bins (Oceanographic conv.).
$                                                         -+
$ - Time in yyyymmdd hhmmss format                         | loop
$                                             -+           |
$ - Point name (C*10), lat, lon, d, U10 and    |  loop     | over
$   direction, current speed and direction     |   over    |
$ - E(f,theta)                                 |  points   | times
$                                             -+          -+
$ 
$ -------------------------------------------------------------------- $
$ End of input file                                                    $
$ -------------------------------------------------------------------- $
\end{verbatim}
\end{footnotesize}
