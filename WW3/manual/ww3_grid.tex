\begin{footnotesize}
\begin{verbatim}
$ -------------------------------------------------------------------- $
$ WAVEWATCH III Grid preprocessor input file                           $
$ -------------------------------------------------------------------- $
$ Grid name (C*30, in quotes)
$
  'TEST GRID (GULF OF NOWHERE)   '
$
$ Frequency increment factor and first frequency (Hz) ---------------- $
$ number of frequencies (wavenumbers) and directions, relative offset
$ of first direction in terms of the directional increment [-0.5,0.5].
$ In versions 1.18 and 2.22 of the model this value was by definiton 0,
$ it is added to mitigate the GSE for a first order scheme. Note that
$ this factor is IGNORED in the print plots in ww3_outp.
$
   1.1  0.04118  25  24  0.
$
$ Set model flags ---------------------------------------------------- $
$  - FLDRY         Dry run (input/output only, no calculation).
$  - FLCX, FLCY    Activate X and Y component of propagation.
$  - FLCTH, FLCK   Activate direction and wavenumber shifts.
$  - FLSOU         Activate source terms.
$
   F T T T F T
$
$ Set time steps ----------------------------------------------------- $
$ - Time step information (this information is always read)
$     maximum global time step, maximum CFL time step for x-y and
$     k-theta, minimum source term time step (all in seconds).
$
    900. 950. 900. 300.
$
$ Start of namelist input section ------------------------------------ $
$   Starting with WAVEWATCH III version 2.00, the tunable parameters
$   for source terms, propagation schemes, and numerics are read using
$   namelists. Any namelist found in the folowing sections up to the
$   end-of-section identifier string (see below) is temporarily written
$   to ww3_grid.scratch, and read from there if necessary. Namelists
$   not needed for the given switch settings will be skipped
$   automatically, and the order of the namelists is immaterial.
$   As an example, namelist input to change SWELLF and ZWND in the
$   Tolman and Chalikov input would be
$
$   &SIN2 SWELLF = 0.1, ZWND = 15. /
$
$ Define constants in source terms ----------------------------------- $
$
$ Stresses - - - - - - - - - - - - - - - - - - - - - - - - - - - - - -
$   TC 1996 with cap    : Namelist FLX3
$                           CDMAX  : Maximum allowed CD (cap)
$                           CTYPE  : Cap type :
$                                     0: Discontinuous (default).
$                                     1: Hyperbolic tangent.
$   Hwang 2011          : Namelist FLX4
$                           CDFAC  : re-scaling of drag
$
$ Linear input - - - - - - - - - - - - - - - - - - - - - - - - - - - -
$   Cavaleri and M-R    : Namelist SLN1
$                           CLIN   : Proportionality constant.
$                           RFPM   : Factor for fPM in filter.
$                           RFHF   : Factor for fh in filter.
$
$ Exponential input  - - - - - - - - - - - - - - - - - - - - - - - - -
$   WAM-3               : Namelist SIN1
$                           CINP   : Proportionality constant.
$
$   Tolman and Chalikov : Namelist SIN2
$                           ZWND   : Height of wind (m).
$                           SWELLF : swell factor in (n.nn).
$                           STABSH, STABOF, CNEG, CPOS, FNEG :
$                                    c0, ST0, c1, c2 and f1 in . (n.nn)
$                                    through (2.65) for definition of
$                                    effective wind speed (!/STAB2).
$   WAM4 and variants  : Namelist SIN3
$                           ZWND    : Height of wind (m).
$                           ALPHA0  : minimum value of Charnock coeffici
$                           Z0MAX   : maximum value of air-side roughnes
$                           BETAMAX : maximum value of wind-wave couplin
$                           SINTHP  : power of cosine in wind input
$                           ZALP    : wave age shift to account for gust
$                       TAUWSHELTER : sheltering of short waves to reduc
$                         SWELLFPAR : choice of swell attenuation formul
$                                    (1: TC 1996, 3: ACC 2008)
$                            SWELLF : swell attenuation factor
$     Extra parameters for SWELLFPAR=3 only 
$                  SWELLF2, SWELLF3 : swell attenuation factors 
$                           SWELLF4 : Threshold Reynolds number for ACC2
$                           SWELLF5 : Relative viscous decay below thres
$                             Z0RAT : roughness for oscil. flow / mean f
$   BYDRZ input         : Namelist SIN6
$                          SINA0    : factor for negative input
$                          SINWS    : wind speed scaling option
$                          SINFC    : high-frequency extent of the
$                                     prognostic frequency region
$
$ Nonlinear interactions - - - - - - - - - - - - - - - - - - - - - - -
$   Discrete I.A.       : Namelist SNL1
$                           LAMBDA : Lambda in source term.
$                           NLPROP : C in sourc term. NOTE : default
$                                    value depends on other source
$                                    terms selected.
$                           KDCONV : Factor before kd in Eq. (n.nn).
$                           KDMIN, SNLCS1, SNLCS2, SNLCS3 :
$                                    Minimum kd, and constants c1-3
$                                    in depth scaling function.
$   Exact interactions  : Namelist SNL2
$                           IQTYPE : Type of depth treatment
$                                     1 : Deep water
$                                     2 : Deep water / WAM scaling
$                                     3 : Shallow water
$                           TAILNL : Parametric tail power.
$                           NDEPTH : Number of depths in for which
$                                    integration space is established.
$                                    Used for IQTYPE = 3 only
$                         Namelist ANL2
$                           DEPTHS : Array with depths for NDEPTH = 3
$   Gen. Multiple DIA   : Namelist SNL3
$                           NQDEF  : Number of quadruplets.
$                           MSC    : Scaling constant 'm'.
$                           NSC    : Scaling constant 'N'.
$                           KDFD   : Deep water relative filter depth,
$                           KDFS   : Shallow water relative filter depth
$                         Namelist ANL3
$                           QPARMS : 5 x NQDEF paramaters describing the
$                                    quadruplets, repeating LAMBDA, MU, 
$                                    Cdeep and Cshal. See examples below
$   Two Scale Approx.   : Namelist SNL4
$                           INDTSA : Index for TSA/FBI computations
$                                    (0 = FBI ; 1 = TSA)
$                           ALTLP  : Index for alternate looping
$                                    (1 = no ; 2 = yes)
$
$ Traditional DIA setup (default):
$
$ &SNL3 NQDEF =  1, MSC =  0.00,  NSC = -3.50 /
$ &ANL3 QPARMS = 0.250, 0.000,  -1.0, 0.1000E+08, 0.0000E+00 /
$
$ GMD3 from 2010 report (G13d in later paper) :
$
$ &SNL3 NQDEF =  3, MSC =  0.00,  NSC = -3.50 /
$ &ANL3 QPARMS = 0.126, 0.000,  -1.0, 0.4790E+08, 0.0000E+00 ,
$                0.237, 0.000,  -1.0, 0.2200E+08, 0.0000E+00 ,
$                0.319, 0.000,  -1.0, 0.1110E+08, 0.0000E+00 /
$
$ G35d from 2010 report:
$
$ &SNL3 NQDEF =  5, MSC =  0.00,  NSC = -3.50 /
$ &ANL3 QPARMS = 0.066, 0.018,  21.4, 0.170E+09, 0.000E+00 ,
$                0.127, 0.069,  19.6, 0.127E+09, 0.000E+00 ,
$                0.228, 0.065,   2.0, 0.443E+08, 0.000E+00 ,
$                0.295, 0.196,  40.5, 0.210E+08, 0.000E+00 ,
$                0.369, 0.226,  11.5, 0.118E+08, 0.000E+00 /
$
$ Nonlinear filter based on DIA  - - - - - - - - - - - - - - - - - - -
$                         Namelist SNLS
$                           A34    : Relative offset in quadruplet
$                           FHFC   : Proportionality constants.
$                           DMN    : Maximum relative change.
$                           FC1-3  : Constants in frequency filter.
$
$ Whitecapping dissipation   - - - - - - - - - - - - - - - - - - - - -
$   WAM-3               : Namelist SDS1
$                           CDIS, APM : As in source term.
$
$   Tolman and Chalikov : Namelist SDS2
$                           SDSA0, SDSA1, SDSA2, SDSB0, SDSB1, PHIMIN :
$                                    Constants a0, a1, a2, b0, b1 and
$                                    PHImin.
$
$   WAM4 and variants   : Namelist SDS3
$                           SDSC1    : WAM4 Cds coeffient
$                           MNMEANP, WNMEANPTAIL : power of wavenumber
$                                    for mean definitions in Sds and tai
$                           SDSDELTA1, SDSDELTA2 : relative weights 
$                                    of k and k^2 parts of WAM4 dissipat
$                           SDSLF, SDSHF : coefficient for activation of
$                              WAM4 dissipation for unsaturated (SDSLF) 
$                               saturated (SDSHF) parts of the spectrum
$                           SDSC2    : Saturation dissipation coefficien
$                           SDSC4    : Value of B0=B/Br for wich Sds is 
$                           SDSBR    : Threshold Br for saturation
$                           SDSP     : power of (B/Br-B0) in Sds
$                           SDSBR2   : Threshold Br2 for the separation 
$                             WAM4 dissipation in saturated and non-satu
$                           SDSC5 : coefficient for turbulence dissipati
$                           SDSC6 : Weight for the istropic part of Sds_
$                           SDSDTH: Angular half-width for integration o
$
$   BYDRZ               : Namelist SDS6
$                          SDSET    : Select threshold normalization spe
$                          SDSA1, SDSA2, SDSP1, SDSP2  :
$                               Coefficients for dissipation terms T1 an
$                       : Namelist SWL6
$                          SWLB1    : Coefficient for swell dissipation
$
$ Bottom friction  - - - - - - - - - - - - - - - - - - - - - - - - - -
$   JONSWAP             : Namelist SBT1
$                           GAMMA   : Bottom friction emprical constant
$
$
$ Surf breaking  - - - - - - - - - - - - - - - - - - - - - - - - - - -
$   Battjes and Janssen : Namelist SDB1
$                           BJALFA  : Dissipation constant (default = 1)
$                           BJGAM   : Breaking threshold (default = 0.73
$                           BJFLAG  : TRUE  - Use Hmax/d ratio only (def
$                                     FALSE - Use Hmax/d in Miche formul
$
$ Dissipation in the ice   - - - - - - - - - - - - - - - - - - - - - - 
$   Generalization of Liu et al. : Namelist SIC2
$                           IC2DISPER  : If true uses Liu formulation wi
$                                        If false, uses the generalizati
$                                        to laminar transition 
$                           IC2TURB    : empirical factor for the turbul
$                           IC2ROUGH   : under-ice roughness length 
$                           IC2REYNOLDS: Re number for laminar to turbul
$                           IC2SMOOTH  : smoothing of transition reprens
$                           IC2VISC    : empirical factor for viscous pa
$
$
$ Scattering in the ice  & creep dissipations- - - - - - - - - - - - - 
$   Generalization of Wiliams et al. : Namelist SIS2
$                           ISC1          : scattering coefficient (defa
$                           IS2BACKSCAT   : fraction of energy back-scat
$                           IS2BREAK      : TRUE  - changes floe max dia
$                                         : FALSE - does not change floe
$                           IS2C1         : scattering in pack ice 
$                           IS2C2         : frequency dependance of scat
$                           IS2C3         : frequency dependance of scat
$                           ISBACKSCAT    : fraction of scattered energy
$                           IS2DISP       : use of ice-specific dispersi
$                           FRAGILITY     : parameter between 0 and 1 th
$                           IS2DMIN       : minimum floe diameter in met
$                           IS2DAMP       : multiplicative coefficient f
$                           IS2UPDATE     : TRUE  - updates the max floe
$                                         : FALSE - updates the max floe
$
$ Dissipation by sea ice
$   Empirical/parametric representations : Namelist SIC4
$                           IC4METHOD    : integer 1 to 7
$                                        : In most cases, additional inp
$                                        :   is required.
$                                        : See examples in /regtests/ww3
$                                        : See also: 1) description in m
$                                        :   and 2) inline documentation
$                                            w3sic4md.ftn
$
$ Triad nonlinear interactions - - - - - - - - - - - - - - - - - - - -
$   Lumped Triad Interaction (LTA) : Namelist STR1 (To be implemented)
$                           PTRIAD1 : Proportionality coefficient (defau
$                           PTRIAD2 : Multiple of Tm01 up to which inter
$                                     is computed (2.5)
$                           PTRIAD3 : Ursell upper limit for computing
$                                     interactions (not used, default 10
$                           PTRIAD4 : Shape parameter for biphase
$                                     computation (0.2)
$                           PTRIAD5 : Ursell number treshold for computi
$                                     interactions (0.01)
$
$ Shoreline reflections - - - - - - - - - - - - - - - - - - - - - - - - 
$   ref. parameters       : Namelist REF1 
$                           REFCOAST  : Reflection coefficient at shorel
$                           REFFREQ   : Activation of freq-dependent ref
$                           REFMAP    : Scale factor for bottom slope ma
$                           REFRMAX   : maximum ref. coeffient (default 
$                           REFFREQPOW: power of frequency 
$                           REFICEBERG: Reflection coefficient for icebe
$                           REFSUBGRID: Reflection coefficient for islan
$                           REFCOSP_STRAIGHT: power of cosine used for 
$                                       straight shoreline
$
$ Bound 2nd order spectrum and free IG - - - - - - - - - - - - - - - - -
$   IG1 parameters       : Namelist SIG1
$                           IGMETHOD  : 1: Hasselmann, 2: Krasitskii-Jan
$                           IGADDOUTP : activation of bound wave correct
$                                       in ww3_outp / ww3_ounp
$                           IGSOURCE  : 1: uses bound waves, 2: empirica
$                           IGSTERMS  :  > 0 : no source term in IG band
$                           IGMAXFREQ : maximum frequency of IG band
$                           IGEMPIRICAL: constant in empirical free IG s
$                           IGBCOVERWRITE: T: Replaces IG spectrum, does
$                           IGSWELLMAX: T: activates free IG sources for
$
$
$ Propagation schemes ------------------------------------------------ $
$   First order         : Namelist PRO1
$                           CFLTM  : Maximum CFL number for refraction.
$
$   UQ/UNO with diffusion : Namelist PRO2
$                           CFLTM  : Maximum CFL number for refraction.
$                           DTIME  : Swell age (s) in garden sprinkler
$                                    correction. If 0., all diffusion
$                                    switched off. If small non-zero
$                                    (DEFAULT !!!) only wave growth
$                                    diffusion.
$                           LATMIN : Maximum latitude used in calc. of
$                                    strength of diffusion for prop.
$
$   UQ/UNO with averaging : Namelist PRO3
$                           CFLTM  : Maximum CFL number for refraction.
$                           WDTHCG : Tuning factor propag. direction.
$                           WDTHTH : Tuning factor normal direction.
$
$   Note that UQ and UNO schemes have no tunable parameters.
$   All tuneable parameters are associated with the refraction
$   limitation and the GSE alleviation.
$
$ Unstructured grids ------------------------------------------------ $
$   UNST parameters       : Namelist UNST 
$                           UGOBCAUTO : TRUE: OBC points are taken from 
$                                       FALSE: OBC points must be listed
$                           UGOBCDEPTH: Threshold ( < 0) depth for OBC p
$                           EXPFSN    : Activation of N scheme
$                           EXPFSPSI  : Activation of PSI scheme
$                           EXPFSFCT  : Activation of FCT scheme
$                           IMPFSN    : Activation of N implicit scheme
$                           IMPTOTAL  : Activation of fully implicit sch
$                           EXPTOTAL  : Turn on implicit refraction (onl
$                           IMPREFRACTION  : Turn on implicit freq. shif
$                           IMPFREQSHIFT  : Turn on implicit freq. shift
$                           IMPSOURCE  : Turn on implicit source terms (
$                           JGS_TERMINATE_MAXITER  : max. Number of iter
$                           JGS_TERMINATE_DIFFERENCE  : terminate based 
$                           JGS_TERMINATE_NORM  : terminate based on the
$                           JGS_USE_JACOBI  : Use Jacobi solver for impt
$                           JGS_BLOCK_GAUSS_SEIDEL  : Use Block Gauss Se
$                           JGS_MAXITER  : max. Number of solver iterati
$                           JGS_PMIN  : % of grid points that do not nee
$                           JGS_DIFF_THR  : implicit solver threshold fo
$                           JGS_NORM_THR  : terminate based on the norm 
$                           SETUP_APPLY_WLV  : Compute wave setup (exper
$                           SOLVERTHR_SETUP  : Solver threshold for setu
$                           CRIT_DEP_SETUP  : Critical depths for setup 
$
$ SMC grid propagation    : Namelist PSMC and default values
$                           CFLTM  : Maximum CFL no. for propagation, 0.
$                           DTIME  : Swell age for diffusion term (s), 0
$                           LATMIN : Maximum latitude (deg) for GCT,   8
$                           RFMAXD : Maximum refraction turning (deg), 8
$                           LvSMC  : No. of refinement level, default 1 
$                           ISHFT  : Shift number of i-index, default 0 
$                           JEQT   : Shift number of j-index, default 0 
$                           NBISMC : No. of input boundary points,    0 
$                           UNO3   : Use 3rd order advection scheme, .FA
$                           AVERG  : Add extra spatial averaging,    .FA
$                           SEAWND : Use sea-point only wind input.  .FA
$ &PSMC DTIME = 39600.0, LATMIN=85.0, RFMAXD = 36.0, LvSMC=3, JEQT=1344 
$
$ Rotated pole ------------------------------------------------------ $
$   Pole  parameters    : Namelist ROTD
$                           PLAT  : Rotated pole latitude
$                           PLON  : Rotated pole longitude
$                           UNROT : Logical, un-rotate directions to tru
$
$    These will be used to derive rotation angle corrections in the
$    model. The corrections are used for rotation of boundary spectra 
$    and for restoring conventional lat/lon orientation of the
$    output spectra, mean direction or any related variables.
$    The PLAT/LON example below is a standard setting for Met
$    Office UK regional models.
$
$ &ROTD PLAT = 37.5, PLON = 177.5, UNROT = .TRUE. /
$
$ Output of 3D arrays------------------------------------------------- $
$ In order to limit the use of memory, arrays for 3D output fiels (i.e. 
$ variables that are a function of both space and frequency, are not 
$ declared, and thus cannot be used, unless specified by namelists.
$ NB: Output of 'first 5' moments E, th1m, sth1m, th2, sth2m allows to e
$  directional spectrum using, e.g. MEM (Lygre&Krogstad 1986). 
$
$ Parameters (integers)   : Namelist OUTS
$ For the frequency spectrum E(f)     
$                          E3D     : <=0: not declared, > 0: declared
$                          I1E3D   : First frequency index of output (de
$                          I2E3D   : Last frequency index of output  (de
$ For the mean direction th1m(f), and spread sth1m(f)     
$                   TH1MF, STH1MF  : <=0: not declared, > 0: declared
$                 I1TH1MF, I1STH1MF: First frequency index of output (de
$                 I2TH1MF, I2STH1MF: First frequency index of output (de
$ For the mean direction th2m(f), and spread sth2m(f)     
$                   TH2MF, STH2MF  : <=0: not declared, > 0: declared
$                 I1TH2MF, I1STH2MF: First frequency index of output (de
$                 I2TH2MF, I2STH2MF: First frequency index of output (de
$ For 2nd order pressure at K=0 (source of microseisms & microbaroms)
$                           P2SF   : <=0: not declared, > 0: declared
$                           I1P2SF : First frequency index of output (de
$                           I2P2SF : Last frequency index of output  (de
$ For the surface Stokes drift partitions (USP)
$                           USSP : First index (default is 1, should alw
$                           IUSSP : Last index (must be <= than NK and s
$                                               between 3 and ~10 with t
$                                               between accuracy and res
$                           STK_WN : List of wavenumbers (size of IUSSP)
$                           e.g.: USSP = 1, IUSSP=3, STK_WN = 0.04, 0.11
$                                 provides 3 partitions of both x & y co
$                                 with a reasonable accuracy for using i
$                                 a climate model. 
$
$ Miscellaneous ------------------------------------------------------ $
$   Misc. parameters    : Namelist MISC
$                           CICE0  : Ice concentration cut-off.
$                           CICEN  : Ice concentration cut-off.
$                           PMOVE  : Power p in GSE aleviation for
$                                    moving grids in Eq. (D.4).
$                           XSEED  : Xseed in seeding alg. (!/SEED).
$                           FLAGTR : Indicating presence and type of
$                                    subgrid information :
$                                     0 : No subgrid information.
$                                     1 : Transparancies at cell boun-
$                                         daries between grid points.
$                                     2 : Transp. at cell centers.
$                                     3 : Like 1 with cont. ice.
$                                     4 : Like 2 with cont. ice.
$                           TRCKCMPR : Logical variable (T/F). Set to F 
$                                      disable "compression" of track ou
$                                      This simplifies post-processing.
$                                      Default is T and will create trac
$                                      output in the traditional manner
$                                      (WW3 v3, v4, v5).
$                           XP, XR, XFILT
$                                    Xp, Xr and Xf for the dynamic
$                                    integration scheme.
$                           IHMAX  : Number of discrete levels in part.
$                           HSPMIN : Minimum Hs in partitioning.
$                           WSM    : Wind speed multiplier in part.
$                           WSC    : Cut of wind sea fraction for
$                                    identifying wind sea in part.
$                           FLC    : Flag for combining wind seas in
$                                    partitioning.
$                           NOSW   : Number of partitioned swell fields
$                                    in field output.
$                           PTM    : Partioning method:
$                                     1 : Default WW3
$                                     2 : Watershedding + wind cutoff
$                                     3 : Watershedding only
$                                     4 : Wind speed cutoff only
$                                     5 : High/Low band cutoff (see PTFC
$                           PTFC   : Cutouf frequency for High/Low band
$                                    partioning (PTM=5). Default = 0.1Hz
$                           FMICHE : Constant in Miche limiter.
$                           STDX   : Space-Time Extremes X-Length
$                           STDY   : Space-Time Extremes Y-Length
$                           STDT   : Space-Time Extremes Duration
$                           P2SF   : ......
$
$ Diagnostic Sea-state Dependent Stress- - - - - - - - - - - - - - - - -
$   Reichl et al. 2014  : Namelist FLD1
$                           TAILTYPE  : High Frequency Tail Method
$                                       0: Constant value (prescribed)
$                                       1: Wind speed dependent
$                                          (Based on GFDL Hurricane
$                                          Model Z0 relationship)
$                           TAILLEV   : Level of high frequency tail 
$                                       (if TAILTYPE==0)
$                                       Valid choices:
$                                       Capped min: 0.001, max: 0.02
$                           TAILT1    : Tail transition ratio 1
$                                       TAILT1*peak input frequency
$                                       is the first transition point of
$                                       the saturation specturm
$                                       Default is 1.25
$                           TAILT1    : Tail transition ratio 2
$                                       TAILT2*peak input frequency
$                                       is the second transition point o
$                                       the saturation specturm
$                                       Default is 3.00
$   Donelan et al. 2012 : Namelist FLD2
$                           TAILTYPE : See above (FLD1)
$                           TAILLEV  : See above (FLD1)
$                           TAILT1   : See above (FLD1)
$                           TAILT2   : See above (FLD1)
$
$ In the 'Out of the box' test setup we run with sub-grid obstacles
$ and with continuous ice treatment.
$
  &MISC CICE0 = 0.25, CICEN = 0.75, FLAGTR = 4 /
  &FLX3 CDMAX = 3.5E-3 , CTYPE = 0 /
$ &SDB1 BJGAM = 1.26, BJFLAG = .FALSE. /
$
$ Mandatory string to identify end of namelist input section.
$
END OF NAMELISTS
$
$ Define grid -------------------------------------------------------- $
$
$ Five records containing :
$
$  1 Type of grid, coordinate system and type of closure: GSTRG, FLAGLL,
$    CSTRG.  Grid closure can only be applied in spherical coordinates.
$      GSTRG  : String indicating type of grid :
$               'RECT'  : rectilinear
$               'CURV'  : curvilinear  
$               'UNST'  : unstructured (triangle-based)
$      FLAGLL : Flag to indicate coordinate system :
$               T  : Spherical (lon/lat in degrees)
$               F  : Cartesian (meters)
$      CSTRG  : String indicating the type of grid index space closure :
$               'NONE'  : No closure is applied
$               'SMPL'  : Simple grid closure : Grid is periodic in the
$                       : i-index and wraps at i=NX+1. In other words,
$                       : (NX+1,J) => (1,J). A grid with simple closure
$                       : may be rectilinear or curvilinear.
$               'TRPL'  : Tripole grid closure : Grid is periodic in the
$                       : i-index and wraps at i=NX+1 and has closure at
$                       : j=NY+1. In other words, (NX+1,J<=NY) => (1,J)
$                       : and (I,NY+1) => (NX-I+1,NY). Tripole
$                       : grid closure requires that NX be even. A grid
$                       : with tripole closure must be curvilinear.
$  2 NX, NY. As the outer grid lines are always defined as land
$    points, the minimum size is 3x3.
$
$ Branch here based on grid type
$
$ IF ( RECTILINEAR GRID ) THEN
$
$  3 Grid increments SX, SY (degr.or m) and scaling (division) factor.
$    If CSTRG='SMPL', then SX is set to 360/NX.
$  4 Coordinates of (1,1) (degr.) and scaling (division) factor.
$
$ ELSE IF ( CURVILINEAR GRID ) THEN
$
$  3 Unit number of file with x-coordinate.
$    Scale factor and add offset: x <= scale_fac * x_read + add_offset.
$    IDLA, IDFM, format for formatted read, FROM and filename.
$      IDLA : Layout indicator :
$                  1   : Read line-by-line bottom to top.
$                  2   : Like 1, single read statement.
$                  3   : Read line-by-line top to bottom.
$                  4   : Like 3, single read statement.
$      IDFM : format indicator :
$                  1   : Free format.
$                  2   : Fixed format with above format descriptor.
$                  3   : Unformatted.
$      FROM : file type parameter
$               'UNIT' : open file by unit number only.
$               'NAME' : open file by name and assign to unit.
$
$    If the above unit number equals 10, then the x-coord is read from t
$    file.  The x-coord must follow the above record.  No comment lines 
$    allowed within the x-coord input.
$
$  4 Unit number of file with y-coordinate.
$    Scale factor and add offset: y <= scale_fac * y_read + add_offset.
$    IDLA, IDFM, format for formatted read, FROM and filename.
$      IDLA : Layout indicator :
$                  1   : Read line-by-line bottom to top.
$                  2   : Like 1, single read statement.
$                  3   : Read line-by-line top to bottom.
$                  4   : Like 3, single read statement.
$      IDFM : format indicator :
$                  1   : Free format.
$                  2   : Fixed format with above format descriptor.
$                  3   : Unformatted.
$      FROM : file type parameter
$               'UNIT' : open file by unit number only.
$               'NAME' : open file by name and assign to unit.
$
$    If the above unit number equals 10, then the y-coord is read from t
$    file.  The y-coord must follow the above record.  No comment lines 
$    allowed within the y-coord input.
$
$ ELSE IF ( UNSTRUCTURED GRID ) THEN
$   Nothing to declare: all the data will be read from the GMESH file 
$ END IF ( CURVILINEAR GRID )
$
$  5 Limiting bottom depth (m) to discriminate between land and sea
$    points, minimum water depth (m) as allowed in model, unit number
$    of file with bottom depths, scale factor for bottom depths (mult.),
$    IDLA, IDFM, format for formatted read, FROM and filename.
$      IDLA : Layout indicator :
$                  1   : Read line-by-line bottom to top.
$                  2   : Like 1, single read statement.
$                  3   : Read line-by-line top to bottom.
$                  4   : Like 3, single read statement.
$      IDFM : format indicator :
$                  1   : Free format.
$                  2   : Fixed format with above format descriptor.
$                  3   : Unformatted.
$      FROM : file type parameter
$               'UNIT' : open file by unit number only.
$               'NAME' : open file by name and assign to unit.
$
$    If the above unit number equals 10, then the bottom depths are read
$    this file.  The depths must follow the above record.  No comment li
$    allowed within the depth input. In the case of unstructured grids, 
$    is expected to be a GMESH grid file containing node and element lis
$
$ ----------------------------------------------------------------------
$ Example for rectilinear grid with spherical (lon/lat) coordinate syste
$ Note that for Cartesian coordinates the unit is meters (NOT km).
$
     'RECT'  T  'NONE'
     12     12
      1.     1.     4.
     -1.    -1.     4.
     -0.1 2.50  10  -10. 3 1 '(....)' 'NAME' 'bottom.inp'
$
  6 6 6 6 6 6 6 6 6 6 6 6
  6 6 6 5 4 2 0 2 4 5 6 6
  6 6 6 5 4 2 0 2 4 5 6 6
  6 6 6 5 4 2 0 2 4 5 6 6
  6 6 6 5 4 2 0 0 4 5 6 6
  6 6 6 5 4 4 2 2 4 5 6 6
  6 6 6 6 5 5 4 4 5 6 6 6
  6 6 6 6 6 6 5 5 6 6 6 6
  6 6 6 6 6 6 6 6 6 6 6 6
  6 6 6 6 6 6 6 6 6 6 6 6
  6 6 6 6 6 6 6 6 6 6 6 6
  6 6 6 6 6 6 6 6 6 6 6 6
$
$ ----------------------------------------------------------------------
$ Example for curvilinear grid with spherical (lon/lat) coordinate syste
$ Same spatial grid as preceding rectilinear example.
$ Note that for Cartesian coordinates the unit is meters (NOT km).
$
$     'CURV'  T  'NONE'
$     12     12
$
$     10 0.25 -0.5 3 1 '(....)' 'NAME' 'x.inp'
$
$   1  2  3  4  5  6  7  8  9 10 11 12
$   1  2  3  4  5  6  7  8  9 10 11 12
$   1  2  3  4  5  6  7  8  9 10 11 12
$   1  2  3  4  5  6  7  8  9 10 11 12
$   1  2  3  4  5  6  7  8  9 10 11 12
$   1  2  3  4  5  6  7  8  9 10 11 12
$   1  2  3  4  5  6  7  8  9 10 11 12
$   1  2  3  4  5  6  7  8  9 10 11 12
$   1  2  3  4  5  6  7  8  9 10 11 12
$   1  2  3  4  5  6  7  8  9 10 11 12
$   1  2  3  4  5  6  7  8  9 10 11 12
$   1  2  3  4  5  6  7  8  9 10 11 12
$
$     10 0.25 0.5 3 1 '(....)' 'NAME' 'y.inp'
$
$   1  1  1  1  1  1  1  1  1  1  1  1
$   2  2  2  2  2  2  2  2  2  2  2  2
$   3  3  3  3  3  3  3  3  3  3  3  3
$   4  4  4  4  4  4  4  4  4  4  4  4
$   5  5  5  5  5  5  5  5  5  5  5  5
$   6  6  6  6  6  6  6  6  6  6  6  6
$   7  7  7  7  7  7  7  7  7  7  7  7
$   8  8  8  8  8  8  8  8  8  8  8  8
$   9  9  9  9  9  9  9  9  9  9  9  9
$  10 10 10 10 10 10 10 10 10 10 10 10
$  11 11 11 11 11 11 11 11 11 11 11 11
$  12 12 12 12 12 12 12 12 12 12 12 12
$
$     -0.1 2.50  10  -10. 3 1 '(....)' 'NAME' 'bottom.inp'
$
$  6 6 6 6 6 6 6 6 6 6 6 6
$  6 6 6 5 4 2 0 2 4 5 6 6
$  6 6 6 5 4 2 0 2 4 5 6 6
$  6 6 6 5 4 2 0 2 4 5 6 6
$  6 6 6 5 4 2 0 0 4 5 6 6
$  6 6 6 5 4 4 2 2 4 5 6 6
$  6 6 6 6 5 5 4 4 5 6 6 6
$  6 6 6 6 6 6 5 5 6 6 6 6
$  6 6 6 6 6 6 6 6 6 6 6 6
$  6 6 6 6 6 6 6 6 6 6 6 6
$  6 6 6 6 6 6 6 6 6 6 6 6
$  6 6 6 6 6 6 6 6 6 6 6 6
$
$ -------------------------------------------------------------
$ SMC grid use the same spherical lat-lon grid parameters 
$     'RECT'  T  'SMPL'
$   1024    704
$ SMC grid base level resolution dlon dlat and start lon lat 
$ 0.35156250  0.23437500   1.
$ 0.17578125 -78.6328125   1.
$
$ Normal depth input line is used to passing the minimum depth
$ though the depth file is not read for SMC grid. 
$  -0.1  10.0  30   -1. 1 1 '(....)' 'NAME' 'SMC25Depth.dat'
$ SMC cell and face arrays and obstruction ratio:
$  32  1  1  '(....)'  'S6125MCels.dat'
$  33  1  1  '(....)'  'S6125ISide.dat'
$  34  1  1  '(....)'  'S6125JSide.dat'
$  31  1.0  1  1 '(...)' 'NAME'  'SMC25Subtr.dat'
$ The input boundary cell file is only needed when NBISMC > 0.
$  35  1  1  '(....)'  'S6125Bundy.dat'
$ Extra cell and face arrays for Arctic part if ARC is selected. 
$  36  1  1  '(....)'  'S6125MBArc.dat'
$  37  1  1  '(....)'  'S6125AISid.dat'
$  38  1  1  '(....)'  'S6125AJSid.dat'
$ Normal land-sea mask file input line is kept but file is not used.
$  39  1  1  '(....)'  'NAME'  'S6125Masks.dat'
$    Boundary cell id list file (unit 35) is only required if boundary 
$    cell number entered above is non-zero.  The cell id number should b
$    the sequential number in the cell array (unit 32) S625MCels.dat.
$
$ If sub-grid information is available as indicated by FLAGTR above,
$ additional input to define this is needed below. In such cases a
$ field of fractional obstructions at or between grid points needs to
$ be supplied.  First the location and format of the data is defined
$ by (as above) :
$  - Unit number of file (can be 10, and/or identical to bottom depth
$    unit), scale factor for fractional obstruction, IDLA, IDFM,
$    format for formatted read, FROM and filename
$
   10 0.2  3 1 '(....)' 'NAME' 'obstr.inp'
$
$ *** NOTE if this unit number is the same as the previous bottom
$     depth unit number, it is assumed that this is the same file
$     without further checks.                                      ***
$
$ If the above unit number equals 10, the bottom data is read from
$ this file and follows below (no intermediate comment lines allowed,
$ except between the two fields).
$
  0 0 0 0 0 0 0 0 0 0 0 0
  0 0 0 0 0 0 0 0 0 0 0 0
  0 0 0 0 0 0 0 0 0 0 0 0
  0 0 0 0 0 0 0 0 0 0 0 0
  0 0 0 0 0 0 0 0 0 0 0 0
  0 0 0 0 0 0 5 0 0 0 0 0
  0 0 0 0 0 0 5 0 0 0 0 0
  0 0 0 0 0 0 4 0 0 0 0 0
  0 0 0 0 0 0 4 0 0 0 0 0
  0 0 0 0 0 0 5 0 0 0 0 0
  0 0 0 0 0 0 5 0 0 0 0 0
  0 0 0 0 0 0 0 0 0 0 0 0
$
  0 0 0 0 0 0 0 0 0 0 0 0
  0 0 0 0 0 0 0 0 0 0 0 0
  0 0 0 0 0 0 0 0 0 0 0 0
  0 0 0 0 0 0 0 0 0 0 0 0
  0 0 0 0 0 0 0 0 5 5 5 0
  0 0 0 0 0 0 0 0 0 0 0 0
  0 0 0 0 0 0 0 0 0 0 0 0
  0 0 0 0 0 0 0 0 0 0 0 0
  0 0 0 0 0 0 0 0 0 0 0 0
  0 0 0 0 0 0 0 0 0 0 0 0
  0 0 0 0 0 0 0 0 0 0 0 0
  0 0 0 0 0 0 0 0 0 0 0 0
$
$ *** NOTE size of fields is always NX * NY                        ***
$
$ Input boundary points and excluded points -------------------------- $
$    The first line identifies where to get the map data, by unit number
$    IDLA and IDFM, format for formatted read, FROM and filename
$    if FROM = 'PART', then segmented data is read from below, else
$    the data is read from file as with the other inputs (as INTEGER)
$
   10 3 1 '(....)' 'PART' 'mapsta.inp'
$
$ Read the status map from file ( FROM != PART ) --------------------- $
$
$ 3 3 3 3 3 3 3 3 3 3 3 3
$ 3 2 1 1 1 1 0 1 1 1 1 3
$ 3 2 1 1 1 1 0 1 1 1 1 3
$ 3 2 1 1 1 1 0 1 1 1 1 3
$ 3 2 1 1 1 1 0 0 1 1 1 3
$ 3 2 1 1 1 1 1 1 1 1 1 3
$ 3 2 1 1 1 1 1 1 1 1 1 3
$ 3 2 1 1 1 1 1 1 1 1 1 3
$ 3 2 1 1 1 1 1 1 1 1 1 3
$ 3 2 1 1 1 1 1 1 1 1 1 3
$ 3 2 1 1 1 1 1 1 1 1 1 3
$ 3 3 3 3 3 3 3 3 3 3 3 3
$
$ The legend for the input map is :
$
$    0 : Land point.
$    1 : Regular sea point.
$    2 : Active boundary point.
$    3 : Point excluded from grid.
$
$ Input boundary points from segment data ( FROM = PART ) ------------ $
$   An unlimited number of lines identifying points at which input
$   boundary conditions are to be defined. If the actual input data is
$   not defined in the actual wave model run, the initial conditions
$   will be applied as constant boundary conditions. Each line contains:
$     Discrete grid counters (IX,IY) of the active point and a
$     connect flag. If this flag is true, and the present and previous
$     point are on a grid line or diagonal, all intermediate points
$     are also defined as boundary points.
$
      2   2   F
      2  11   T
$
$  Close list by defining point (0,0) (mandatory)
$
      0   0   F
$
$ Excluded grid points from segment data ( FROM != PART )
$   First defined as lines, identical to the definition of the input
$   boundary points, and closed the same way.
$
      0   0   F
$
$   Second, define a point in a closed body of sea points to remove
$   the entire body of sea points. Also close by point (0,0)
$
      0   0
$
$ Sedimentary bottom map if namelist &SBT4 SEDMAPD50 = T
$
$     22  1. 1 1 '(f10.6)' 'NAME' 'SED.txt'
$
$ Output boundary points --------------------------------------------- $
$ Output boundary points are defined as a number of straight lines,
$ defined by its starting point (X0,Y0), increments (DX,DY) and number
$ of points. A negative number of points starts a new output file.
$ Note that this data is only generated if requested by the actual
$ program. Example again for spherical grid in degrees. Note, these do
$ not need to be defined for data transfer between grids in the multi
$ grid driver.
$
     1.75  1.50  0.25 -0.10     3
     2.25  1.50 -0.10  0.00    -6
     0.10  0.10  0.10  0.00   -10
$
$  Close list by defining line with 0 points (mandatory)
$
     0.    0.    0.    0.       0
$
$ -------------------------------------------------------------------- $
$ End of input file                                                    $
$ -------------------------------------------------------------------- $
\end{verbatim}
\end{footnotesize}
