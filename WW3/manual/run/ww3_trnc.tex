\vsssub
\subsubsection{Track output NetCDF post-processor} \label{sec:ww3trnc} 
\vsssub

\proddefH{ww3\_trnc}{w3trnc}{ww3\_trnc.ftn}
\proddeff{Input}{ww3\_trnc.nml}{Namelist configuration file.}{10} (App.~\ref{sec:config202})
\proddefa{ww3\_trnc.inp}{Traditional configuration file.}{10} (App.~\ref{sec:config201})
\proddefa{track\_o.ww3}{Raw track output data.}{11}
\proddeff{Output}{standard out}{Formatted output of program.}{6}
\proddefa{*.nc}{NetCDF file.}{}

\vspace{\baselineskip} 
\noindent
This post-processor will convert the raw track output data to a NetCDF file. 
The output NetCDF file contains the following variables :

\begin{list}{$\bullet$}{\itemsep 0mm \parsep 0mm}
\item time in days since 1990-01-01, Julian days at UTC time.
\item frequencies of each frequency bin. (radian)
\item frequencies of lower band of each frequency bin. (radian)
\item frequencies of upper band of each frequency bin. (radian)
\item frequencies width of each frequency bin. (radian)
\item directions of each direction bin. (oceanographic convention)
\item latitudes and longitudes along time dimension. (degree)
\end{list}

\noindent
For each output point varying in time and position, the following records are printed :
\begin{list}{$\bullet$}{\itemsep 0mm \parsep 0mm}
\item track name (32 characters)
\item the entire spectrum. (m2 s rad-1)
\item water depth (m), current and wind u and v components (m s-1),
friction velocity (m s-1), air-sea temperature difference (degree
centigrade).
\end{list}

\pb
