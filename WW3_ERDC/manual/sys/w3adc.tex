\vssub
\subsection{~The preprocessor} \label{sec:w3adc}
\vssub

The \ws\ source code files are not ready to use {\fortran} files; mandatory
and optional program options still have to be selected, and test output may be
activated\footnote{~Exceptions are some modules that are not originally part
  of \ws, like the exact interaction modules. Such modules with the extension
  {\file .f} or {\file .f90} bypass the preprocessor and get copied to the
  work directory with the {\file .f} extension.}. Compile level options are
activated using `switches'. The arbitrary switch '{\F swt}' is included in the
\ws\ files as comment of the form {\F !/swt}, where the switch name {\F swt}
is followed by a space or by a '{\F /}'. If a switch is selected, the
preprocessor removes the comment characters, thus activating the corresponding
source code line. If '{\F/}' follows the switch, it is also removed, thus
allowing the selective inclusion of hardware-dependent compiler directives
etc. The switches are case sensitive, and available switches are presented in
\para\ref{sec:switches}.

Files which contain the switch {\F c/swt} can be
found by typing \command{find\_switch '!/SWT'} A list of all switches included
in the \ws\ files can be obtained by typing \command{all\_switches}

\pb
\noindent
Traditionally, only one switch can be used per line of source code. However,
as of {\ww} version 7.00, multiple switches (up to 4) can be set on a single
line. For the line to be included in the compilation step all switches must
be activated. Multiple switches must be chained together at the beginning of
the line, with no spaces in between. Each switch must start with '{\F !/}' and
may optionally be terminated with '{\F /}'. This allows for the handling of
circumstances where a source code line needs to be included only if a
particular combination of switches is set. For example, to enable a line only
if the '{\F /OMPH}' and '{\F /T1}' switches are set:

\begin{footnotesize}
\begin{verbatim}
!/OMPH!/!T1     ! Line included only if OMPH and T1 activated
\end{verbatim}
\end{footnotesize}

\pb
%\vspace{\baselineskip}
\noindent
Pre-processing is performed by the program {\code w3adc}. This program is
found in the file {\file w3adc.f}, which contains a ready to compile
{\fortran} source code and a full documentation\footnote{~Presently still in
  fixed-format {\fortran}-77.}. Various properties of {\code w3adc} are set in
{\F parameter} statements in {\file w3adc.f}, i.e., the maximum length of
switches, the maximum number of include files, the maximum number of lines in
an include file and the line length. {\code w3adc} reads its `commands' from
standard input. An example input file for {\code w3adc} is given in
Fig.~\ref{fig:w3adc}. Line-by-line, the input consists of

\begin{list}{$\rightarrow$}{\itemsep 0mm \parsep 0mm}
\item	Test indicator and compress indicator.
\item	File names of the input and output code.
\item	Switches to be turned on in a single string (see
          \para\ref{sec:switches}).
\item   Additional lines with include files can be given, but these are no
        longer used in the automated compile system.
\end{list}

% fig:w3adc

\setlength{\unitlength}{0.1mm}

\begin{figure}

\begin{center} {\code \begin{tabular}{|cllc|} \hline
 & & &  \\
 & \multicolumn{2}{l}{0 1}        & \\ 
 & constants.ftn' & constants.f'  & \\
 & \multicolumn{2}{l}{'F90 NOGRB SHRD PR3 UQ FLX2 LN1 ST2 STAB2}& \\
 & \multicolumn{2}{l}{\strut\hspace{5mm} NL1 BT1 DB1 MLIM TR0 BS0 XX0 WNX1 WNT1 CRX1 CRT1 }& \\
 & \multicolumn{2}{l}{\strut\hspace{5mm} O0 O1 O2 O3 O4 O5 O6 O7 O11 O14'}& \\
 &               &             & \\ \hline
\end{tabular} } \end{center} 
\caption{Example input for {\F w3adc}.} \label{fig:w3adc}

\botline
\end{figure}


\noindent
A test indicator 0 disables test output, and increasing values increase the
detail of the test output. A compress indicator 0 leaves the file as is. A
compress indicator 1 results in the removal of all comment lines indicated by
'{\F !}', except for empty switches, i.e., lines starting with '{\F !/}'. A
compress indicator 2 results in the subsequent removal of all comments.
Comment lines are not allowed in this input file. The above input for {\F
w3adc} is read using free format. Therefore quotes are needed around
strings. Echo and test output is send to the standard output device. To
facilitate the use of the preprocessor, several \unix\ scripts are provided
with \ws\ as discussed in \para\ref{sec:comp}. Note that compiler directives
are protected from file compression by defining them using a switch.

