\vsssub
\subsubsection{The input field preprocessor } \label{sec:ww3prep}
\vsssub

\proddefH{ww3\_prep}{w3prep}{ww3\_prep.ftn}
\proddeff{Input}{ww3\_prep.inp}{Traditional configuration file.}{10} (App.~\ref{sec:config051})
\proddefa{mod\_def.ww3}{Model definition file.}{11}
\proddefa{'user input'\opt}{See example below.}{user}
\proddeff{Output}{standard out}{Formatted output of program.}{6}
\proddefa{level.ww3\opt}{Water levels file.}{12}
\proddefa{current.ww3\opt}{Current fields file.}{12}
\proddefa{wind.ww3\opt}{Wind fields file.}{12}
\proddefa{ice.ww3\opt}{Ice fields file.}{12}
\proddefa{data0.ww3\opt}{Assimilation data (`mean').}{12}
\proddefa{data1.ww3\opt}{Assimilation data (`1-D spectra').}{12}
\proddefa{data2.ww3\opt}{Assimilation data (`2-D spectra').}{12}

\vspace{\baselineskip} 
\vspace{\baselineskip} 

\noindent 
Note that the optional output files are specific to {\file ww3\_shel} and
{\file ww3\_multi}, but are not processed by the actual wave model
routines. These files are consequently not needed if the wave model routines
are used in a different shell or in an integrated program. However, the
routines reading and writing these files are system-independent and could
therefore be used in customized applications of the basic wave model. The
reading and writing of these files is performed by the subroutine {\F w3fldg}
({\file w3fldsmd.ftn}). For additional documentation and file formats
reference if made to this routine.
\pb
