\vsssub
\subsubsection{Grid Integration} \label{sub:ww3gint}
\vsssub
\proddefH{ww3\_gint}{w3gint}{ww3\_gint.ftn}
\proddeff{Input}{ww3\_gint.inp}{Traditional configuration file.}{10} (App.~\ref{sec:config111})
\proddefa{mod\_def.*}{Model definition files in \ws\ format for base and target grids}{20}
\proddefa{out\_grd.*}{Gridded field files in \ws\ format for base grids}{30+}
\proddeff{Output}{standard out}{Formatted output of program.}{6}
\proddefa{out\_grd.*}{Gridded field files in \ws\ format for target grid}{30+}

\vspace{\baselineskip}
\noindent
This post processor program takes field data from several overlapping grids
and produces a unified output file. The different model definition and field
output files are identified by the unique identifier associated with each
specific grid. At this moment the program works with curvilinear and
rectilinear grids. A weights file {\file WHTGRIDINT.bin} is written 
that can be read in subsequent runs using identical origin-destination grids, 
saving substantial time in cases using large number of input grids and/or 
high-resolution target grids.

\vspace{\baselineskip}
\vspace{\baselineskip}
\noindent
Note that this program can be used in concert with the grid splitting program
{\file ww3\_gspl}, and that {\file ww3\_gspl.sh} has an option to produce a
template input file for his program (see \para\ref{sub:ww3gspl}).

\pb
