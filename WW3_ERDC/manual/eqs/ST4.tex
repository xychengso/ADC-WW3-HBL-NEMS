\vsssub
\subsubsection{~$S_{\mathrm{in}} + S_{\mathrm{ds}}$: Saturation-based dissipation} \label{sec:ST4}
\vsssub

\opthead{ST4}{\ws}{F. Ardhuin, J.-F. Filipot \& L. Romero}

\noindent 
This family of parameterizations uses a positive part of the wind input taken
from WAM cycle 4 with an ad hoc reduction of $u_\star$, implemented in
order to allow a balance with a saturation-based dissipation that uses different options for 
a cumulative term. There are three main options for defining the saturation and the cumulative term. Chosing one or the other is done with the  {\F SDSBCHOICE} parameter, with  {\F SDSBCHOICE=1} for \cite{art:Aea10},  {\F SDSBCHOICE=2} for \cite{Filipot&Ardhuin2012}, and {\F SDSBCHOICE=3} for \cite{Romero2019}. That last options uses a saturation that is defined from the local spectral density, and thus gives zero dissipation for directions where the threshold is not reached, leading to much broader directional spectra. Also the stronger bimodality is achieved by having a strong modulation effect as a cumulative term. 

Many other adjustments can be made by changing the namelist parameters. A few successful combinations 
are given by tables \ref{tab:ST4_parSIN} and \ref{tab:ST4_parSDS}, with results described by \citep{art:RA13,art:SAG16}. 
Further calibration to any particular wind field should be done for best performance. Guidance for this is given by \cite{Stopa2018}. 
%We also note that the particular 
%set of parameters T400 corresponds to setting IPHYS=1 in the ECWAM code cycle 45R2, with a few differences 
%related to the fact that precomputed stress tables have now been removed from ECWAM. 

\vsssub
\textbf{Wind input and swell dissipation} 
\vsssub


The reduction of $u_\star$ in
eq. (\ref{eq:SinWAM4}) is obtained by replacing it with $u_\star '(k)$ defined for each
frequency as

\begin{equation}
\left(u_\star '\right)^2=\left|u_\star^2 \left(\cos \theta_u, \sin
\theta_u \right) - \left|s_u\right| \int_0^k \int_0^{2 \pi}
\frac{S_{in}\left(k',\theta \right)}{C}  \left(\cos \theta, \sin
\theta \right)  {\mathrm d} k' \mathrm d
\theta,\label{ustarp}\right|
\end{equation}

\noindent 
where the sheltering coefficient $\left|s_u\right|\sim 1$ can be used to tune
the stresses at high winds, which would be largely overestimated for
$s_u=0$. For $s_u > 0$ this sheltering is also applied within the diagnostic
tail in eq. (\ref{eq:tauhfint}), which requires the estimation of a
3-dimensional look-up table for the high frequency stress, the third parameter
being the energy level of the tail.

The {\code STAB3} switch, described above for use with {\code ST3}, may also be used with {\code ST4}. If {\code STAB3} is used, the air-sea  temperature differences should be provided by the user, e.g. using {\file ww3\_prep}.

The swell dissipation parameterization of \cite{art:ACC09} is activated by
setting $s_1$ to a non-zero integer value, and is given by a combination of
the viscous boundary layer value,

\begin{equation}
\cS_{\mathrm{out,vis}}\left(k,\theta\right) = - s_5 \frac{\rho_a}{\rho_w}\left\{ 2 k \sqrt{2
\nu \sigma}\right\}  N \left(k,\theta\right) , \label{eq:Dore}
\end{equation}

\noindent
with the turbulent boundary layer expression 
\begin{equation}
\cS_{\mathrm{out,tur}} \left(k,\theta\right) = - \frac{\rho_a}{\rho_w}\left\{  16 f_e
\sigma^2 u_{\mathrm{orb},s} / g \right\}
 N\left(k,\theta\right),  \label{eq:swell_turb}
\end{equation}

\noindent
giving the full term 
\begin{equation}
\cS_{\mathrm{out}} \left(k,\theta\right) = r_{vis} \cS_{\mathrm{out,vis}}\left(k,\theta\right)  + 
 r_{tur} \cS_{\mathrm{out,tur}}\left(k,\theta\right),  
 \label{eq:swell_comb}
\end{equation}

\noindent
where the two weights $ r_{\mathrm{vis}} $ and $r_{\mathrm{tur}}$ are defined from 
a modified  air-sea boundary layer significant Reynolds number $\mathrm{Re} = 2
u_{\mathrm{orb},s} H_s / \nu_{a}$ 

\begin{eqnarray}
 r_{\mathrm{vis}} &=& 0.5 (1- \tanh((\mathrm{Re}-\mathrm{Re}_{c})/s_7), \\
 r_{\mathrm{tur}}&=& 0.5 (1+ \tanh((\mathrm{Re}-\mathrm{Re}_{c})/s_7) .
\end{eqnarray}
The significant surface orbital velocity is defined by
\begin{equation} u_{\mathrm{orb},s} = 2 \left [  \int \!\!\!\! \int
      \sigma^3 \: N(k,\theta) \: dk d\theta \right ] ^{1/2}
      \: . \label{eq:ub_orbs} \end{equation}

\noindent 
The first equation (\ref{eq:Dore}) is the linear viscous decay by
\cite{art:Dore78}, with $\nu_a$ the air viscosity and $s_5$ is an $O(1)$
tuning parameter. A few tests have indicated that a threshold Re$_{c}=2 \times
10^5 \times (4~\mathrm{m}/H_s)^{(1-s_6)}$ provides reasonable result with
$s_6=0$, although it may also be a function of the wind speed, and we have no
explanation for the dependence on $H_s$.  With $s_6=1$, a constant threshold
close to $2 \times 10^5$ provides similar -- but less accurate -- results.


\begin{landscape}
\begin{table} 
\begin{center} 
%\begin{tabular}{|l|c|c|c|c|c|c|c|c|} \hline \hline
%Par.         &  WWATCH var.       & namelist & T471    & T471f       & T400/$I_{\mathrm{phys}}=1$    & T405          & T500         & T601     \\
%\hline
%  $z_u$ &  ZWND                       & SIN4 & 10.0    & 10.0        & 10.0           & 10.0          & 10.0         & 10.0        \\
%  $\alpha_0$ &  ALPHA0                & SIN4 & 0.0095  & 0.0095      & \textbf{0.0062}& 0.0095        &  0.0095      & 0.0095      \\
%  $\beta_{\mathrm{max}}$ & BETAMAX    & SIN4 & 1.43    &\textbf{1.33}&\textbf{1.42}   & \textbf{1.55} &\textbf{1.52} & \textbf{2.0}\\
%  $p_{\mathrm{in}}$ &  SINTHP         & SIN4 & 2       & 2           & 2              & 2             &  2           & \textbf{1}  \\
%  $z_\alpha$ &  ZALP                  & SIN4 & 0.006   & 0.006       & 0.008          & 0.006         &0.006         & 0.006       \\
%  $s_u$ &  TAUWSHELTER                & SIN4 & 0.3     & 0.3         & \textbf{0.25}  & \textbf{0.0}  &\textbf{1.0}  & \textbf{0.5}\\
%  $s_1$ &  SWELLF                     & SIN4 & 0.66    & 0.66        & 0.66           & 0.8           &  0.8         & 0.66        \\
%  $s_2$ &  SWELLF2                    & SIN4 & -0.018  & -0.018      &-0.018          & -0.018        &  -0.018      &  -0.018     \\
%  $s_3$ &  SWELLF3                    & SIN4 &  0.022  &  0.022      & 0.022          &\textbf{0.015} &\textbf{0.015}&  0.022      \\
%  $\mathrm{Re}_c$ &  SWELLF4          & SIN4 &$1.5\X^5$& $1.5\X^5$   & $1.5\X^5$      &$\mathbf{10^5}$&$\mathbf{10^5}$&  $1.5\X^5$ \\
%  $s_5$ &  SWELLF5                    & SIN4 & 1.2     & 1.2         & 1.2            & 1.2           &  1.2         & 1.2         \\
%  $s_6$ &  SWELLF6                    & SIN4 & 0.      & 0.          & \textbf{1.0}   & 0.            & 0.           & 0.          \\
%  $s_7$ &  SWELLF7                    & SIN4 &3.6$\X^5$&3.6$\X^5$    & 3.6$\X^5$      &\textbf{0.0}   &\textbf{0.0}  & 3.6$\X^5$   \\
%  $z_r$ &  Z0RAT                      & SIN4 & 0.04    & 0.04        & 0.04           & 0.04          &  0.04        &   0.04      \\
%  $z_{0,\max}$ &  Z0MAX               & SIN4 & 1.002   & 1.002       & 1.002          &\textbf{0.002} &  1.002       &  1.002      \\
%\hline
%\end{tabular}  
\begin{tabular}{|l|c|c|c|c|c|c|c|} \hline \hline
Par.         &  WWATCH var.       & namelist & T471    & T471f       & T405          & T500         & T601     \\
\hline
  $z_u$ &  ZWND                       & SIN4 & 10.0    & 10.0        & 10.0          & 10.0         & 10.0        \\
  $\alpha_0$ &  ALPHA0                & SIN4 & 0.0095  & 0.0095      & 0.0095        &  0.0095      & 0.0095      \\
  $\beta_{\mathrm{max}}$ & BETAMAX    & SIN4 & 1.43    &\textbf{1.33}& \textbf{1.55} &\textbf{1.52} & \textbf{2.0}\\
  $p_{\mathrm{in}}$ &  SINTHP         & SIN4 & 2       & 2           & 2             &  2           & \textbf{1}  \\
  $z_\alpha$ &  ZALP                  & SIN4 & 0.006   & 0.006       & 0.006         &0.006         & 0.006       \\
  $s_u$ &  TAUWSHELTER                & SIN4 & 0.3     & 0.3         & \textbf{0.0}  &\textbf{1.0}  & \textbf{0.5}\\
  $s_1$ &  SWELLF                     & SIN4 & 0.66    & 0.66        & 0.8           &  0.8         & 0.66        \\
  $s_2$ &  SWELLF2                    & SIN4 & -0.018  & -0.018      & -0.018        &  -0.018      &  -0.018     \\
  $s_3$ &  SWELLF3                    & SIN4 &  0.022  &  0.022      &\textbf{0.015} &\textbf{0.015}&  0.022      \\
  $\mathrm{Re}_c$ &  SWELLF4          & SIN4 &$1.5\X^5$& $1.5\X^5$   &$\mathbf{10^5}$&$\mathbf{10^5}$&  $1.5\X^5$ \\
  $s_5$ &  SWELLF5                    & SIN4 & 1.2     & 1.2         & 1.2           &  1.2         & 1.2         \\
  $s_6$ &  SWELLF6                    & SIN4 & 0.      & 0.          & 0.            & 0.           & 0.          \\
  $s_7$ &  SWELLF7                    & SIN4 &3.6$\X^5$&3.6$\X^5$    &\textbf{0.0}   &\textbf{0.0}  & 3.6$\X^5$   \\
  $z_r$ &  Z0RAT                      & SIN4 & 0.04    & 0.04        & 0.04          &  0.04        &   0.04      \\
  $z_{0,\max}$ &  Z0MAX               & SIN4 & 1.002   & 1.002       &\textbf{0.002} &  1.002       &  1.002      \\
\hline
\end{tabular}



 \end{center}
\caption{Parameter values for T471, T471f, T405, T500, and T601 source 
term parameterizations that can be reset via the {\F SIN4} namelist. 
Please note that the names of the variables only apply to the namelists. In the
source term module the names are slightly different, with a doubled first
letter, in order to differentiate the variables from the pointers to these
variables, and the SWELLFx are combined in one array SSWELLF. Values highlighted in bold are
different from the default values set by ww3\_grid.} \label{tab:ST4_parSIN}
\end{table}
\end{landscape}

TEST471 generally provides the best results at global scale when using ECMWF winds,
with the only serious problem being a low bias for $H_s > 8$~m.  TEST451f
corresponds to a retuning for CSFR wind reanalysis from NCEP/NCAR
\citep{art:CFSRR10}, and has almost no bias all the way to $H_s =
15$~m. Simulations and papers prepared before March 2012, used slightly
different values, {\it e.g.} TEST441 and TEST441f can be recovered by setting SWELLF7 to 0, and 
TEST471 also used $s_u=1$ and a few other adjustements (see manual of version 4.18).
TEST405 is slightly superior for short fetches, and TEST500 is intermediate in
terms of quality but it also includes depth-induced breaking in the same
formulation, and thus may be more appropriate for depth-limited conditions.


Eq. (\ref{eq:swell_turb}) is a parameterization for the
nonlinear turbulent decay. When comparing model results to observations, it
was found that the model tended to underestimate large swells and overestimate
small swells, with regional biases. This defect is likely due, in part, to
errors in the generation or non-linear evolution of theses swells. However, it
was chosen to adjust $f_e$ as a function of the wind speed and direction,

\begin{equation}
f_e = s_1 f_{e,GM} + \left[\left|s_3\right| + s_2 \cos
(\theta-\theta_u)\right]u_\star / u_{\mathrm{orb}},\label{fevar}
\end{equation}

\noindent 
where $f_{e,GM}$ is the friction factor given by Grant and Madsen's
(1979)\nocite{art:GM79} theory for rough oscillatory boundary layers without a
mean flow, using a roughness length adjusted to $r_z$ times the roughness for
the wind $z_1$. The coefficient $s_1$ is an $O(1)$ tuning parameter, and the
coefficients $s_2$ and $s_3$ are two other adjustable parameters for the
effect of the wind on the oscillatory air-sea boundary layer. When $s_2 < 0$,
wind opposing swells are more dissipated than following swells. Further, if
$s_3 > 0$, $\cS_{out}$ is applied to the entire spectrum and not just the
swell.


\vsssub
\textbf{Wave breaking and ocean turbulence effects} 
\vsssub


The dissipation term is parameterized from the wave spectrum saturation, following the general ideas of \cite{art:Phi85}.
%which were initially explored in a numerical modeling framework by \cite{art:AB03}. 
The saturation spectrum is 
\begin{equation}
B\left(k,\theta\right)= \sigma k^3  N(k,\theta) \label{defB},
\end{equation}
and corresponds to a dimensionless form of the surface elevation spectrum. In general, going from the spectral space to the physical space requires 
integrating the saturation over a finite spectral band in wavenumber and direction to compute the breaking probability 
and then deconvolve this integral to obtain a spectral dissipation rate. Because such operations would be too time consuming we have 
implemented three approaches. One uses on integration over directions only \citep{art:Aea10}, while the second uses an integration  over frequency 
bands \citep{Filipot&Ardhuin2012}, and the last actually uses the local value of $B$ without integration at all. Using one or the other version activated by the namelist parameter {\F  SDSBCHOICE  }.

Because the directional wave spectra were too narrow when using a
saturation spectrum integrated over the full circle \citep{art:AL06},
\citet{art:Aea10} restricted over a sector of half-width $\Delta_\theta$,
\begin{equation}
B'\left(k,\theta\right)=
\int_{\theta-\Delta_\theta}^{\theta+\Delta_\theta} \sigma k^3 cos^{\mathrm{sB}}\left(\theta-
\theta^{\prime}\right) N(k,\theta^{\prime}) \mathrm d
\theta^{\prime} \label{defBofkprime}.
\end{equation}
As a result, a sea state with two systems of same energy but opposite
direction will typically produce less dissipation than a sea state with all
the energy radiated in the same direction.

Based on recent analysis by \cite{Guimaraes2018} and \cite{Peureux&al.2019}, this saturation is enhanced by a factor $M_L$ that represents 
the effect of long waves on short waves 
\begin{equation}
M_l(k,\theta)=1+M_\theta \sqrt{\mathrm{mss}(k,\theta)} + N_\theta \sqrt{\mathrm{nss}(k,\theta)} \label{defFACSAT}.
\end{equation}
where $M_\theta$ is twice the modulation transfer function for short wave steepness, with 
$M_\theta=8$ when following the simplified theory by \cite{art:LHS60} and using the root mean square enhancement of $B$ over a 
long wave cycle. $N_\theta$ is an additional straining factor due to the instability of the wave action envelope of short waves 
propagating in the direction close to that of the long wave \citep{Peureux&al.2019}. The squared slopes $\mathrm{mss}(k,\theta)$ is 
the mean square slope in direction $\theta$, wheras $\mathrm{nss}(k,\theta)$ is a slope of long waves propagating in a narrow window $\pm \delta_\theta$, 
around the short wave direction $\theta$.

We finally define our dissipation term as the sum of the saturation-based term
and a cumulative breaking term $S_{\mathrm{bk,cu}}$,
\begin{eqnarray}
\cS_{ds}(k,\theta)& =&  \sigma
 \frac{C_{\mathrm{ds}}^{\mathrm{sat}}}{B^2_r} \left[ \delta_d
\max\left\{ M_l(k,\theta) B\left(k\right) -
B_r,0\right\}^2 \right.
\nonumber \\
  & & +  \left(1-\delta_d \right) \left. \max\left\{ M_L(k,\theta) B'\left(k,\theta \right)- B_r
 ,0\right\}^2\right]N(k,\theta)  \nonumber \\
 & & + \cS_{\mathrm{bk,cu}}(k,\theta) + \cS_{\mathrm{turb}}(k,\theta) \label{Sds_all}.
\end{eqnarray}
where
\begin{equation}
B\left(k \right)=\max\left\{B'(k,\theta), \theta \in [0,2
\pi[\right\} \label{defBof}.
\end{equation}
The combination of an isotropic part (the term that multiplies $ \delta_d$)
and a direction-dependent part (the term with $1-\delta_d$) was intended to
allow some control of the directional spread in resulting spectra.

The cumulative breaking term $\cS_{\mathrm{bk,cu}}$ represents the smoothing
of the surface by big breakers with celerity $C'$ that wipe out smaller waves
of phase speed $C$. Due to uncertainties in the estimation of this effect in
various observations, we use the theoretical model of
\cite{art:Aea09}. Briefly, the relative velocity of the crests is the norm of
the vector difference, $\Delta_C =\left|\mathbf{C}-\mathbf{C}'\right|$, and
the dissipation rate of short wave is simply the rate of passage of the large
breaker over short waves, i.e. the integral of $\Delta_C \Lambda(\mathbf{C})
d\mathbf{C}$, where $\Lambda (\mathbf{C}) d\mathbf{C}$ is the length of
breaking crests per unit surface that have velocity components between $C_x$
and $C_x+dC_x$, and between $C_y$ and $C_y+dC_y$ \citep{art:Phi85}.  Here
$\Lambda$ is inferred from breaking probabilities. Based on Banner et
al. (2000, figure 6, $b_T=22
\left(\varepsilon-0.055\right)^2$)\nocite{art:BBY00}, and taking their
saturation parameter $\varepsilon$ to be of the order of $1.6
\sqrt{B'(k,\theta)}$, the breaking probability of dominant waves is
approximately
\begin{equation}
P=56.8\left(\max\{\sqrt{B'(k,\theta)}-\sqrt{B'_r},0\}\right)^2.\label{PBanner}
\end{equation}
However, because they used a zero-crossing analysis, for a given wave scale,
there are many times when waves are not counted because the record is
dominated by another scale: in their analysis there is only one wave at any
given time.  This tends to overestimate the breaking probability by a factor
of 2 \citep{art:FAB10}, compared to the present approach in which it is 
considered that several waves (of different scales) may be present at the same place and
time. This effect is corrected simply dividing $P$ by 2.


With this approach the spectral density of crest length (breaking or not) per
unit surface $l(\mathbf{k})$ such that $\int l(\mathbf{k}) \mathrm{d}k_x
\mathrm{d}k_y$, we take
\begin{equation}
l(\mathbf{k})= 1/(2\pi^2 k),
\end{equation}
and the spectral density of breaking crest length per unit surface is
$\Lambda(\mathbf{k})=l(\mathbf{k})P(\mathbf{k})$.  


Finally the last option for the ST4 implementation of a saturation-based dissipation is as generalization 
of the parameterization by \citet{Romero2019}, giving some flexibility in the definition of the modulation 
term, this is activated by setting  {\F  SDSBCHOICE  = 3}. In this case the  breaking crest density for wavenumber vector  
$\mathbf{k}$ is a function of the saturation only for the same wavenumber, $B(\mathbf{k})$, with no integration in direction 
or frequency (which would be problematic if applied ot monochromatic wave spectra), but a strong modulation $M_L$  by long waves
and a wind-dependent correction $M_W$
\begin{equation}
\Lambda (\mathbf{k}) = \frac{1}{k} \exp \left(-\frac{B_{\mathbf{r}}}{M_l(\mathbf{k})  B(\mathbf{k})}\right) M_L(\mathbf{k}) M_W(k).
\end{equation}
where the wind factor is, 
\begin{equation}
 M_W(k)=\left( 1+ D_W  \max(1,k/k_0) \right) / (1+D_W),
\end{equation}
where $k_0=g (3/ 28 u_\star)^2$, and $D_W$ was adjusted to 0.9 when using the DIA, and $D_W=2$ for the exact nonlinear interaction.
The long wave modulation is either applied to $B$ (if $N_\theta > 0$) and takes the form $ M_l(k,\theta)$ given above, 
or, as done in \cite{Romero2019} the modulation is applied to  $\Lambda$ and takes the form, 
\begin{equation}
 M_L(k,\theta)=\left( 1+ M_\Lambda \sqrt{\mathrm{mss}'(k,\theta)}  \right)^{1.5}.
\end{equation}
In that expression, the cumulated slope $\mathrm{mss}'$ is either $\mathrm{mss}(k,\theta)$ or, if $N_\theta=0$, it is forced to have a $\cos^2 (\theta-\theta_w)$ variation, 
with $\theta_w$ the direction of the mean dominant waves (which is independant of $k$). In \cite{Romero2019} this strong $\cos^2$ directional dependency of $\Lambda$ is key for producing 
strong bimodal spectra, and it may be a compensation for the isotropic dissipation rate. Indeed, the full source term for {\F  SDSBCHOICE  = 3} reads 
\begin{eqnarray}
\cS_{ds}(k,\theta)=  \frac{C_{\mathrm{ds}}^{\mathrm{sat}} (\sqrt{B(k)}-\sqrt{B_T} )^2.5}{g^2} \Lambda (k,\theta) c^5 + \cS_{\mathrm{bk,cu}}(k,\theta) + \cS_{\mathrm{turb}}(k,\theta) \label{Sds_all}. \nonumber \\
\end{eqnarray}
where it should be noted that the saturation used in the dissipation rate $b=C_{\mathrm{ds}}^{\mathrm{sat}} (\sqrt{B(k)}-\sqrt{B_T} )^2.5/{g^2}$ is integrated over direcitons. and uses a true threshold $B_T$ that is different from the $B_r$ in the expression of $\Lambda$. 
Also, the generalization of \cite{Romero2019} allows to use a modulation of $B$ or $\Lambda$ and offers different options for the directional distribution of this modulation. Also, this breaking term can be combined with the cumulative term of \cite{art:Aea10} and the wave-turbulence interaction term of \citet{art:AJ06}. 

For all three choices of {\F SDSBCHOICE}, the additional cumulative and wave-turbulence interaction terms are computed in the same way. 
Assuming that any breaking
wave instantly dissipates all the energy of all waves with frequencies higher
than a factor $r_{\mathrm{cu}}$ or more, the cumulative dissipation rate is
simply given by the rate at which these shorter waves are taken over by larger
breaking waves, times the spectral density, namely 
\begin{equation}
\cS_{\mathrm{bk,cu}}(k,\theta) = -C_{\mathrm{cu}}  N \left(k,\theta\right) \int_{f' < r_{\mathrm{cu}} f } \Delta_C \Lambda(\mathbf{k'}) \mathrm{d\mathbf{k'}},
\label{Sds_cu1}
\end{equation}
where $r_{\mathrm{cu}}$ defines the maximum ratio of the frequencies of long
waves that will wipe out short waves.  This gives the source term,
\begin{eqnarray}
\cS_{\mathrm{bk,cu}}(k,\theta) &=&  \frac{-14.2 C_{\mathrm{cu}}}{\pi^2}  N \left(k,\theta\right)
 \nonumber \\
& &\int_0^{ r^2_{\mathrm{cu}} k }\int_0^{2\pi}
\max \left\{\sqrt{B(f',\theta')}-\sqrt{B_r},0\right\}^2
\mathrm{d}\theta' \mathrm{d}k'.
\label{Sds_sat_isotropic}
\end{eqnarray}
We shall take $r_{\mathrm{cu}}=0.5$, and $C_{\mathrm{cu}}$ is a tuning
coefficient expected to be of order 1, which also corrects for errors in the
estimation of $l$.


Finally, the wave-turbulence interaction term of \cite{art:TB02} and \cite{art:AJ06},
is given by

\begin{equation}
\cS_{\mathrm{ds}}^{\mathrm{TURB}}\left(k,\theta\right) = - 2
C_{\mathrm{turb}} \sigma \cos(\theta_u - \theta) k \frac{\rho_a
u_\star^2}{g \rho_w}  N\left(k,\theta\right) .
\end{equation}

\noindent
The coefficient $C_{\mathrm{turb}}$ is of order 1 and can be used to adjust for
ocean stratification and wave groupiness.

All relevant source term parameters can be set via the namelists {\F SIN4} and {\F SDS4}
to yield parameterizations TEST441b, TEST405, both described by
\cite{art:Aea10} or TEST500 described by \cite{art:FA12} (see Tables \ref{tab:ST4_parSIN} and \ref{tab:ST4_parSDS}). Please note that the
DIA constant $C$ has been slightly adjusted in TEST441b, $C=2.5\times
10^7$. TEST441f corresponds to a re-tuned wind input formulation when using
NCEP/NCAR winds.

\begin{landscape}
\begin{table} \begin{center} 
%\begin{tabular}{|l|c|c|c|c|c|c|c|c|} \hline \hline
%Par.                               &  WWATCH var.  & namelist  & T471        & T400/$I_{\mathrm{phys}}=1$& T405             & T500         & T601        &  T700      \\
%\hline
% 				    & SDSBCHOICE    & SDS4      &  1          &     1                     &   1              & \textbf{2}   & 1           & \textbf{3} \\  
%%  $p$                             &  WNMEANP      & SDS4      & 0.5         & 0.5                       & 0.5              &  0.5    \\
%%  $p_{\mathrm{tail}}$             &  WNMEANPTAIL  & SDS4      & 0.5         & 0.5                   & 0.5 &  0.5 \\
%  $f_{\mathrm{FM}}$                &  FXFM3        & SDS4      & 2.5         & 2.5                       & 2.5              &\textbf{9.9}  & 5           &  \textbf{20} \\
%                                   & SDSC1         & SDS4      & 0           & 0                         & 0                &\textbf{1.0}  & 0           &              \\
%  $C_{\mathrm{ds}}^{\mathrm{sat}}$ & SDSC2         & SDS4      &$-2.2\X^{-5}$&$-2.2\X^{-5}$              &$-2.2\X^{-5}$     &\textbf{0.0}  &$-2.2\X^{-5}$& \textbf{-3.8}   \\
%  $C_{\mathrm{ds}}^{\mathrm{BCK}}$ & SDSBCK        & SDS4      & 0           & 0                         & 0                &\textbf{0.185}& 0           &   \\
%  $C_{\mathrm{ds}}^{\mathrm{HCK}}$ & SDSHCK        & SDS4      & 0           & 0                         & 0                &\textbf{1.5}  & 0           & \\
%  $\Delta_\theta$                  & SDSDTH        & SDS4      & 80          & 80                        & 80               & 80           &  80         & \\
%  $\delta_\theta$                  & SDSSTRAINA    & SDS4      & 0           & 0                         & 0                & 0            & \textbf{15} &    \\
%  $M_\theta$                       & SDSSTRAIN     & SDS4      & 0           & 0                         & 0                & 0            & \textbf{10} &    \\
%  $N_\theta$                       & SDSSTRAIN2    & SDS4      & 0           & 0                         & 0                & 0            & \textbf{20} & \textbf{0}    \\
%  $B_r$                            & SDSBR         & SDS4      & 0.0009      & 0.0009                    &\textbf{0.00085}  & 0.0009       & 0.0009      & \\
%  $C_{\mathrm{cu}}$                & SDSCUM        & SDS4      & -0.40344    & \textbf{0.0}              & \textbf{0.0}     &-0.40344      &-0.40344     & \\
%  ${\mathrm{s_B}}$                 & SDSCOS        &SDS4       & 2.0         &  2.0                      & \textbf{0.0}     & 2.0          & 2.0         & \\
%  $B_0$                           & SDSC4         & SDS4      & 1.0         & 1.0                       & 1.0              & 1.0          & 1.0          & \\
%  $p^{\mathrm{sat}}$               & SDSP          & SDS4      & 2.0         & 2.0                       & 2.0              & 2.0          & 2.0         & \\
%  $C_{\mathrm{turb}}$              & SDSC5         & SDS4      & 0.0         & 0.0                       &  0.0             & 0.0          & \textbf{1.0}&  \\
%  $\delta_d$                       & SDSC6         & SDS4      & 0.3         & 0.3                       &  0.3             & 0.3          &0.3          & \\
%  $C$                              & NLPROP        & SNL1      & $2.5\X^7$   & $\mathbf{2.7\X^7}$        &$\mathbf{2.7\X^7}$& $2.5\X^7$    & $2.5\X^7$   & \\
% \hline \hline
%\end{tabular}  
\begin{tabular}{|l|c|c|c|c|c|c|c|} \hline \hline
Par.                               &  WWATCH var.  & namelist  & T471        & T405             & T500         & T601        & T700  \\
\hline
 				   & SDSBCHOICE    & SDS4      &  1          &     1            & \textbf{2}   & 1           & \textbf{3}  \\  
  $f_{\mathrm{FM}}$                &  FXFM3        & SDS4      & 2.5         & 2.5              &\textbf{9.9}  & 5           & \textbf{20} \\
%                                   & SDSC1         & SDS4      & 0           & 0                &\textbf{1.0}  & 0           &    \\
  $C_{\mathrm{ds}}^{\mathrm{sat}}$ & SDSC2         & SDS4      &$-2.2\X^{-5}$&$-2.2\X^{-5}$     &\textbf{0.0}  &$-2.2\X^{-5}$&  \textbf{-3.8}  \\
  $C_{\mathrm{ds}}^{\mathrm{BCK}}$ & SDSBCK        & SDS4      &            &                   & 0.185        &             & \\
  $C_{\mathrm{ds}}^{\mathrm{HCK}}$ & SDSHCK        & SDS4      &            &                   &  1.5         &             & \\
  $\Delta_\theta$                  & SDSDTH        & SDS4      & 80          & 80               & 80           &  80         & \\
  $\delta_\theta$                  & SDSSTRAINA    & SDS4      & 0           & 0                & 0            & \textbf{15} &  0  \\
  $M_\theta$                       & SDSSTRAIN     & SDS4      & 0           & 0                & 0            & \textbf{10} &  0  \\
  $N_\theta$                       & SDSSTRAIN2    & SDS4      & 0           & 0                & 0            & \textbf{20} &  \textbf{0}   \\
  $B_r$                            & SDSBR         & SDS4      & 0.0009      &\textbf{0.00085}  & 0.0009       & 0.0009      &  \textbf{0.005}  \\
  $B_r$                            & SDSBT         & SDS4      &             &                  &              &             &    0.0011     \\
  $C_{\mathrm{cu}}$                & SDSCUM        & SDS4      & -0.40344    & \textbf{0.0}     &-0.40344      &-0.40344     & \textbf{0.0} \\
  ${\mathrm{s_B}}$                 & SDSCOS        &SDS4       & 2.0         & \textbf{0.0}     & 2.0          & 2.0         &     \\
  $p^{\mathrm{sat}}$               & SDSP          & SDS4      & 2.0         & 2.0              & 2.0          & 2.0         &     \\
  $C_{\mathrm{turb}}$              & SDSC5         & SDS4      & 0.0         &  0.0             & 0.0          & \textbf{1.0}& 0.0 \\
  $\delta_d$                       & SDSC6         & SDS4      & 0.3         &  0.3             & 0.3          &0.3          & \\
  $ M_W$			   & SDSMWD	   & SDS4      &            &                   &              &             & 0.9 \\
  $ M_\Lambda$			   & SDSFACMTF	   & SDS4      &            &                   &              &             & 400 \\
  $C$                              & NLPROP        & SNL1      & $2.5\X^7$   &$\mathbf{2.7\X^7}$& $2.5\X^7$    & $2.5\X^7$   & $2.5\X^7$ \\
 % $p_{\mathrm{mss}}$              &  SPMSS         & SDS4      &            &                   &             &             &  0.5 \\
 \hline \hline
\end{tabular}  
\end{center}

\caption{Same as Table \ref{tab:ST4_parSIN}, for the {\F SDS4} and {\F SNL1}
namelists. Bold values are different from the default values set by 
 ww3\_grid. Values are omitted when the SDSBCHOICE makes them not used. Note that \cite{Romero2019} suggests using  $ M_W=2$ when NL2 is used.	 } \label{tab:ST4_parSDS}
%\botline
\end{table}
\end{landscape}
